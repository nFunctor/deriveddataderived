\documentclass{article}
%\usepackage{baskervald}
%\usepackage[french,russian]{babel}
\usepackage[utf8]{inputenc}
\linespread{1.07}
\usepackage{graphicx}
\usepackage{hyperref}
\usepackage{tikz-cd}
\title{Derived -- Data -- Derived}
\author{}

\begin{document}
\thispagestyle{empty}

\maketitle

\noindent Derived. It should be natural for you by now.




Have you noticed how complicated life has become? How even some of our daily routines are of such complexity they escape our views? Social phenomena look like chains that one is having a hard time tracing. Even start with yourself. If you have a hobby, will it be easy to explain it to a stranger? Maybe, you know the details of historic white tie attire, or a bombload of a 1950s fighter-bomber jet, and that may even make for an interesting conversation, but how long will it take to relate? Have you perhaps attended a therapist, and how many times have you returned to the same events in your childhood, from different angles through different chains of reasoning?
When you laugh at those obscure meme videos, can you always explain the references to others without breaking the laugh? Do you want to, have to?


Have you perhaps been well educated to do some technical, complex job, and you know other educated people to whom you cannot really talk about the specifics of it? Even fishing takes some time explaining, how about doing the same with operating a collider or writing a proper backend? Look at the markets and their financial derivatives, a crucial example in a way, where banks only think they know that a given mortgage debt is traced to the person X, who is in fact lending their home to Y \cite{BGSHT}, and perhaps this Y pays the rent by selling their work - the older kind of debt - to some murky internet platform. And while the financial institutions still think they know how to measure some of those processes, the mass of the data escapes them.

The financial crash and the recession of 2008 still influence our lives, and there is no sign that such events will not
happen again (if anything, there are signs to the contrary \cite{FORBES}). Those familiar with the details
\cite{BGSHT} would argue, independently of whether current economical system is doomed to have crises,
that the mortgage crash happened due to the system lacking information, arguably not caring by design.
Curiously, the neoliberal market system that we find ourselves in was meant to prevent such carelessness
from happening: its opponent in the face of centralised planning of USSR was compared to an ineffective computer \cite{DWIT}.

And many of those from ex-USSR will confirm to you that central planning employed there was indeed often ineffective in
allocation of resources,
and certainly just as conservative as the big banks when it came to caring about the actual economy to the
lives of individuals and changing its algorithms as a result. In fact, arguably, the so-called
``socialist'' economy of the USSR never went past the capiltalist stage, just that now the
whole country functioned like a big corporation  \cite{CLIFF}. Most of the
population had no real say in how planning targets were set. Finally, the USSR economy was
not about distribution of products according to their kind \cite{NEURATH}, but instead
stayed rather faithful to the ideas of money and price. This model never escaped the fallback
of using a simple quantity to measure all things in a world such complex as ours.


Today, as every second passes, we are aware more and more about the lives of individuals and
greater economic agents, thanks to the digitalised \emph{economy of data}. We do not need anymore to imagine
a dystopian scenario when all our movements, tastes and abilities are known to someone watching. From
NSA scandals to the routine of tech companies, we find ourselves in the situation when this has already
happenned: the cyberpunk is the material of our age.
And if it is so that the data about us is abundant, we may as well attempt to make something
out of it.

Indeed, the importance of data analysis has already
been understood by many investors, but the development of machine learning and data analysis algorithms
is always geared towards the goal of rising profit margins, of reducing our picture of the economy to
something overly simplistic. But why? Why would one need to
organise the economy in such a way, to tie all the data we have to profits whereas we
could simply attempt to operate directly with the data itself? The usual asnwer is that
of course this is too complicated, not to mention various questions of incentives.

In the last few years, partially fueled by the worsening situation in income inequality, ecology and politics,
there have been more and more attempts to see if there is a different way of doing things, of witnessing a different
way of economic and social organisation.
All the protests around the globe, but also philosophers, sociologists, some economists have sensed something in the air.
%The arrival of the term “postmodernism” signified that a sufficient amount of knowledge data was generated in their fields to draw sufficient amount of links to feel that this is happening (and even to analyze them using some rigour \cite{MORAVA}).
However, various attempts to deal with the challenge of formulating a new paradigm seemed so far lacking something.
For example, a relatively known Accelerate Manifesto \cite{SRNICEK} is a clear sign that there is indeed search happening for the next step of our collective existence, but even the first phrase of this manifesto cannot escape saying “more modern”, reflecting in my opinion a lack of tools in the traditional ways of studying the society. The point that I want to make in this essay is that the tools are actually there, and perhaps we are headed not to ``more modernity'', but to \emph{derived modernity}.

Below I will attempt to explain this idea. To fix a definition, a “derived” phenomenon is a situation in which one deals with chains of relations of incredible complexity. Often such
relations can have relations between themselves. I argued above that such derived phenomena
are encountered in our daily lives. Let me now give a few examples from my
professional field, where we managed to conquer the mountain of infinite relations.
%Here is, how.

\subsection*{Derived Mathematics}

Mathematics, a headache discipline for many a student and professor, is itself a very derived way of knowing the objective reality. We started by interacting with the reality as primitive animals, then we acted on it with our tools, with our production. Our heads were becoming filled with more and more concepts. As our tools were becoming better, we even questioned the perception of already familiar things such as time and space. And while this was happening, we also developed mathematics. As everyone who tried to do it knows, it is objectively real without existing in the form of a chair. It is both encoded into our society via language, culture and books, but also constantly draws its force from the great unknown. It also exists as data, today. And while one can never find sufficiently many apples on the planet to form the set of all natural numbers, we know that we don’t have to do so to understand number theory, since we know how to abstract, derive our understanding of infinity in an already practically pertinent way.

But mathematics went much further than infinities of count.

There was analysis \cite{ANALYSIS}, which was about trying to make our naive feeling of an uninterrupted, continuous process more clear. It is not clear to many students and even adult mathematicians as of today, remaining an active field of research, often applied (and if you find this example of mine too complex, maybe the summary at the end of this subsection can help). Its simplicity is devilish, as the real line and the plane are inhabited with such richness that one can stumble onto the names of Cantor \cite{CANTOR}, Peano \cite{PEANO} and Mandelbrot
\cite{MANDELBROT}. So to deal with this at least for a little, we introduced the derivative, an abstraction of the deviation, the relative difference between two close points, inherent to the infinitely small region around them. Its inspiration was the mechanical notion of speed.

\begin{center}
\includegraphics[scale=0.045]{Tangent_to_a_curve.jpg}

\textit{The derivative represented via the notion of tangent line: the line that one can draw through a couple of points ``ultra-close'' to each other. Image from \cite{DERIVATIVE}.}
\end{center}

 And we learned how to study many functions by measuring their derivatives, by interacting with them, by writing equations on those derivatives, that can be relatively simple but completely mysterious in practice. Those equations remain under active investigation, be it hydrodynamics or general relativity.

So, in the chaos of real number variety, we wrote a relation of derivative, and understood something. But after that, we went further.

From drawing triangles on the surface of our planet, to understanding the shape of the planet, and then pehaps of the space-time itself, that was the process from which we learned geometry and topology, the abstraction of the notion of a space. We were not happy to just study single spaces anymore, we started to understand relations between them: how a spiral can be viewed as an infinte covering of a circle, how having holes in a donut corresponds to tracing paths in it, or formally, mapping circles into it \cite{HATCHER}.

\begin{center}
\includegraphics[scale=0.7]{UniversalCoveringOfCircle}

\textit{The spiral of real numbers can be mapped onto a circle by the means of projecting each point vertically. This allows to better understand various mathematical properties of the circle. Taken from \cite{NLABCIRC}.}
\end{center}


 And we discovered unintuitive facts even about things that we can witness with our eyes, like the usual two-sphere, roughly the shape of the surface of our globe itself. And establishing relations was key in understanding how to do this: to analyze the data of a mysterious space we can see if we can “draw” (“map”) familiar spaces into it for example, and then infer something about it. Many things are still yet to be understood in topology (even the higher-dimensional spheres are a mystery), but by establishing a scheme of relations in the form of algebraic topology \cite{HATCHER, MAY} helped us a lot.

Algebraic topology studied not only how spaces relate between themselves, but how to see spaces from a different angle altogether. One simple way was just to count how many “disjoint pieces” a space has, or “how many holes”, but the homology theories go beyond. And so, while I cannot recount it in detail here, we do have many such “different points of view”. So not only spaces are related to each other, but there are also different ways to relate all the spaces to quantifiable invariants. The idea of a derivative as a relation between the two “super-closest” points is still here, but things have increased drastically in complexity.

We dealt with it, by introducing the notion of a \emph{category} \cite{EILMAC,MACLANE}. It is our abstraction of the notion that a bunch of “objects” of study, together with various “relations” (strictly called morphisms) between them. These relations can be ``composed'', as per this illustration:
$$
X \stackrel f \longrightarrow Y, \, \, Y \stackrel g \longrightarrow Z \, \, \, \, \, \Rightarrow \, \, \, \, \, X \stackrel{g \circ f}{\longrightarrow} Z
$$
\begin{center}
\textit{A diagrammatic illustration of one of the basic properties in category: having a relation (morphism) $f$ from an object named $X$ to an object named $Y$, and a relation from $Y$ to some object $Z$ means there is a unique relation (usually called $g \circ f$) from $X$ to $Z$, just like in the logic of propositions.}
\end{center}

We know how to do mathematics of such entities, describe their internal properties, study various properties of objects of a category as a result of its interactions with all the other objects (including oneself).
Yoneda’s lemma \cite{MACLANE} even formalises the statement that an object is essentially fully described by all the possible interactions with other objects of its kind. It may appeal, without a doubt, to those familiar with the dialectic materialist concept of the identity of the social and the individual, something that can be phrased more simply by saying that we live in a society and are not free from it, but in fact find ourselves in an interaction with it. Our mother tongue is not something that we have chosen, for example, and in many cultures (say, the British) the way one speaks can often be a sign of their social origins.

The devil is in the details. Knowing all relations to all objects may be an impossible task. People started counting number of holes in spaces for a reason: it was a simpler thing to do. And as I said, it corresponds to some procedure that knows how to relate spaces to numerical invariants. All spaces are assigned some sort of an invariant (say a set of matrices – tables filled with numbers)\footnote{For those familiar with the subject, I simplify here, meaning in reality groups, abelian groups or vector spaces as per usual.}, and relations between such spaces amount to certain relations between these assigned matrices. What we are doing here in terms of categories is that we are drawing a relation between the category of spaces and a category of numerical invariants. And such relations can be many. We call them functors, derived from the term of functions.

We went even further. We now have categories, and ways to relate them. Meaning categories themselves become objects of the category of categories, with functors as relations. And when we looked closer, we actually discovered that given two categories $\mathcal C, \mathcal D$ and two relations-functors $F: \mathcal C \to \mathcal D$ and $G: \mathcal C \to \mathcal D$, there are in fact relations possible between $F$ and $G$! We thus can consider categories, relations between them, and relations between relations. We called such thing a 2-category. Are there many possible 2-categories? Yes. Can one define relations between them? Yes. Are there relations between such relations etc? Yes. Can one continue, into 3, 4, 5, infinity categories? Yes \cite{BAEZ, NLABINFTY}.

\begin{center}
\begin{tikzcd}
  A \arrow[r, bend left, "f", ""{name=U,inner sep=1pt,below}]
  \arrow[r, bend right, "g"{below}, ""{name=D,inner sep=1pt}]
  & B
  \arrow[Rightarrow, from=U, to=D, "\alpha"]
\end{tikzcd}

\smallskip

\textit{In the $2$-category above, relations $f,g$ between $A$ and $B$ turn out to be objects themselves, related by a ``two-relation'' (``two-morphism'') $\alpha$.}
\end{center}


%And along this way, one will notice that what we dealt with before had infinity-categories within them all the way.
Thus in an $n$-category, one has that relations are in fact themselves objects. Are not they after all? For a glimpse of an example, consider a space that you can visualise. We will now try to form a $2$-category. Take the points of your space as objects, draw paths between them, call them relations. There can indeed be many paths from point A to point B. But some of them can be related, for example if you never had to pass through a ‘hole’ in your space if you intuitively deformed one path to another:

\begin{center}
\includegraphics[scale=0.5]{paths}

\textit{Here is an illustration. Take a disk and consider the black region in the centre removed. Consider paths starting at a point to the right of the ``black hole''. We can deform the path $\alpha$  into $\beta$, but we cannot deform any of them into the path $\gamma$ without passing through the black region. Modified image from \cite{NAKAHARA}.}
\end{center}

All possible deformations (formally called homotopies) are themselves in fact paths “in the space of paths”, just like paths were “points”, or “objects in the category of paths”. You can go on deriving this example up to infinity. And in fact, for lots of purposes, understanding this “infinity”-category with points, paths, paths between paths and beyond is just as good as understanding the original space itself. Relating spaces to infinity-categories like that is yet another functor from the category of spaces to the category of infinity-categories.

How did we not get lost in such a disaster? The feeling that a non-specialist gets from it may explain the length of education that one needs to understand what I am saying in details, let alone master it. But many years of work of many mathematicians permitted us to establish an understanding of “higher category theory”, and apply it to get “derived” versions of familiar mathematical concepts, to illuminate problems old and new. While the apparatus at hand is a bit esoteric, it is rigorous and sufficiently-well founded. We have shown, thanks to many, that the conquest of the derived is possible \cite{LURIE}.

Let me revisit my story and try to underline what point I was trying to make. The  derivative of real functions expresses the relation between two points infinitely close to each other. It is a relation. In category theory, we deal with the relations in the abstract. Two objects of a category can have many relations between them, but there are always ways to “compose” such relations in a unique way.
In a higher category, relations are themselves “objects” of sorts: one has relations between relations, relations between relations between relations and so on.
So, a higher category has objects, relations, 2-relations, 3-relations and so on.
Because of this, having a relation  $X \to Y$ and $Y \to Z$ means that one can get a relation of composition $X \to Z$, but not in a unique way. Example of this is the path infinity-category, where one can travel different (but often similar) paths between two points. Those paths can be related by a 2-relation, and not a single one. In return, those 2-relations can be related (or not related) by 3-relations. And this tower continues.
As a result, in all practical situations, asking a question “how to compose a given chain of 1-relations” is a bit meaningless. We control these chains via other techniques (“handling the coherences”). Traditionally this field is called derived, or homotopical, or higher category theory.

\subsection*{Data is The Link}

This is where we can return back to the rest of our life. Mathematics (relatively ancient compared to the stories of above) powers a lot of it already, as it has been doing since the millenia of history \cite{CHILDE}, and was applied to social phenomena all along, including some fairly recent attempts \cite{MORAVA}. Physics also stressed the importance of interactions rather than objects themselves in the quantum theory of fields \cite{WEINBERG}. So, where are we now? Are these ideas I mentioned above finding their way into economy?
%One has a computer boom after all, with all the recent momentum of artificial intelligence, data analysis and online platforms.
Current machine learning and data analysis often rely on mathematics that is dozens if not centuries of years old, the mathematics that we learned to explain to the computer. But as we speak at this very moment, we are learning how to do better, and use some fairly recent mathematics to analyze data.

The AI that we have now may seem mathematically unappealing at first, but it is very effective for how simple it is. It can already probably make it through a complicated calculus course by the means of simplifying symbolic expressions \cite{LAMPCHART}. Some other people, including the author, are trying to explain simple category theory to the neural networks. The latter are already able to understand how to ``compose'' relations
in a category, and who knows what they can infer from looking at millions and millions of categorical examples.
Another attempt is to program the infinity-categories associated to spaces in order to make capable proof assistants, in the form of homotopy type theory \cite{HOTT}. And while I am not expert enough to comment on the theory of AI, the attempts to make progress on “deriving” the current AI in the form of General AI (that some study as ``the AI relating other AIs'') are underway.
Independently of the questions of AI, we also learned to associate spaces to various data sets, and to study their structure
using the methods of algebraic topology, a subject known as persistent homology. In
fact, one can use the ``shapes'' associated to data to detect cancer \cite{LAWSON} or
even study various connections in the brain \cite{PETRI}.

\begin{center}
\includegraphics[scale=0.51]{noisy_circle_simplexes_and_persistent_homology.png}

\textit{An application of algebraic topology to study data, known as persistent homology. Roughly speaking, one studies a set of data by drawing gradually various shapes around it, and reducing them to numerical invariants. Image taken from \cite{BUNCH}}.
\end{center}


All mentioned applications are recent, merely scratching the surface of the box of possibilities. What is happening globally is that we are “pumping” our mathematical knowledge into the vision of the machines, the same machines that operate our economy, and hence, if you are materialist, influence our society. Computers, data help us to transfer the abstract discoveries of the few into the fabric of our being, so that the many will feel it too. In return, it will, of course, change our science and mathematics itself.

A dialectic materialist observation of history states that our development as a species is about witnessing the reality, interacting with it, producing, be it objects or concepts. Objects produced influence the reality in return, concepts do too, influencing ourselves and then our society. The same observation then claims that history is about the dynamics of this process  \cite{MARX,MARXENGCPE}, noting that in some periods our ways of production are not in harmony with our relations between ourselves. Most recent significant attempts at harmonisation were the all too familiar events in XVIII century France and XX century Russia, and there is no way to deny how they changed the look of the world that came after. It is probably possible by now to see where I am headed with my long-winded argument. The economy is drawing from science via machine learning and data analysis,
but the relations of production and the relations of our society are inherited from much less advanced times.
The issue of harmonisation between this new economy and our social relations will arise again. At some point, even more of the abstract mathematics that I mention will be drawn into the economy in its full splendor, and will also allow for a better understanding of our social processes. At this moment, we will find ourselves
in a situation resembling a chainge of social formations. Pehaps we are finally ready to step outside the
boundaries of ``post-modernism''.
The revolution seems inevitable, and it will be derived.

What form it may take I can only guess. If I were naive and used my mathematical education to argue, I would say that we have to take all that happened before, and make it into derived. The class that is in progressive relations with the production forces will realise its derived proletarian character. It will have to fight against the (arguably already derived) mechanism of repression in the form of ``derived nobility'', by forming an organisation that will be derived, of sorts. I can vouch for the feeling of what those things could be, but not for much more. Whatever form it takes, refusing to see or accept this coming change may well turn out to be the cause of greater cataclysms.

Once liberated, the derived economy of data will be aware of us, our relations, and how to classify us by relating our processes to other processes. It will not need currency and markets of the old capitalist ways, as currency and finances are simply ineffective when faced against the derived data. The latter will simply know better what we actually can do and do want, and will suggest how to organise our action accordingly. Analyising the data will lead to the possibility of arranging distribution of products in a way that makes the monetary approximations disappear.

Even the question of incentives can be approached differently, through a combination of
``suggestions'' so common in current machine-learning and videogame-like ``achievements''.
Not only the derived economy of data would clarify what the role
of each person in the economy is, it would also use the methods of
derived data analysis to take us towards a more meaningful and harmonious economy.

Let me iterate that the derived economy of data would not be a veiled state-capitalist “centralised planning” in the USSR sense. It would avoid monetary approximations and would prioritise learning the use-values from what it sees, producing in accord with the need. For now, it is difficult for a single person to even guess the exact shape of the algorithms that this economy would employ,
even though I am confident that such a formulation is possible. For instance, it is a known phenomenon that neural
networks tend to cluster the data while analyisng it \cite{CLUSTERISATION}, something that can potentially lead to the emergence of use-value
categories within the algorithm aware of the whole economy.

Perhaps this algorithm would adapt a planinfication on many levels in accordance with the different ways the data presents itself,
then solving a multi-actor problem to understand all possible outputs, and presenting the population with the variety of possible outcomes. In this direction, it would perhaps even employ a derived version of Input-Output matrices \cite{KANTOROVICH} at some point. The population would then
make its choice by voting via the AI-secretaries, an idea already described in \cite{DWIT}, and this vote would then decide the production and distribution for the next time period. One could even imagine that the plan would be dynamically updated,
taking into account the relative development of production. In such a way, the notion of say ``five-year'' plan would be unnecessary, as the automated vote could take place on a much more regular basis. A dynamic update of the distribution
is crucial to not end up in situations of urgency for the essentials: provided the AI-secretaries and the people behind them
are equally sound in the system, many major consumption crises could be avoided, something that may become of crucial importance in the period of transition to the economy of data.

As is apparent, the derived economy of data would be about drawing more and more links between science and the social, and eventually about understanding the links between those links, deriving them all. For this reason, it would make the closed-source and data-secrecy often unnatural, as the whole point would be about establishing as many links as possible, knowing as much as possible, with respect to that data and those who produce it. Although on the initial stages, it will of course have to deal with the neo-feudal fragmentation imposed by tech giants and startups, and the arguably well-founded fear of the individuals about their privacy.


\subsection*{Making The Leap}

This all may sound overly optimistic, of course. How to make the change happen? Let
us sketch a few more guesses.

The supposed ``derived proletariat'' would at first
recognise its character as a class. It is already the case that previously prestigious
professions related to academia and tech are experiencing severe pressure, from
the precarity and overexploitation state of academia \cite{ACADANON} to the suppression
of workers organising attempts at big tech \cite{GOOG}. There is a general \cite{GRAEBER}
realisation of the uselessness of your own activity in the world of BS jobs. A new economical crisis may
well accelerate those developments: the neo-feodalist financial elite has all the reasons to
be afraid of a genuine economy of data. We are thus witnessing the development of the ``derived''
part of proletariat, that may well produce a concentrate of its revolutionary energy in the form of an organisation. Would this organisation be ``the chosen one''? Not really, not unless it
establishes a coherent and dialectic interaction with the other layers of the working population,
be it traditional working class or uberised proletariat (``the platformitariat''). After all,
``derived'' is omnipresent and inclusive: traditional industrial proletariat is derived too.
The resulting synthetic entity
of an organisation could well have potential to convince a majority of the population to
follow its lead in the transformation of the society.

Should this orgnaisation take the form of a political party? Perhaps, but again, it
can use science and technology to ensure its functioning. Not only one could apply
operational research \cite{BEER} to keep track of a proper, democratic yet coherent,
unified and efficient structure. One could in fact attempt to automate it. The text \cite{DWIT}
spoke of the idea of automated advisors to help a participative economy, but one can also
introduce such advisors in a political party, to automate the processes of debate and
vote. One can go further and make a global machine network that would act as a
suggestion system and would analyze the situation in order to facilitate decision-making
that makes the derived revolution into reality. Our popular culture has been obssessed
with AI's as benevolent dictators managing the economy, but an AI that analyzes
data is simply us as a whole, in a systematic, scientific way. And as the experience
of the degeneration of the Soviet revolution shows, managing the politics is just as
important as managing economics: even if the former is a ``concentrate'' of the latter \cite{LENIN},
they are still in a dialogue. And it seems that if we cannot escape politics, it would
be rather desirable to automate the politicians. Transparency is key here: the questions
of ``Alignment problem'' of the AI may become very relevant, but not unsolvable.

The sketch I describe should not be elaborated upon in the abstract. The author is of course
not free from the context of France and Western Europe: developed economies with significant, but not
overwhelming (like the US) presence of financial capital. And as my data economy argument
shows, the financial capital is already becoming irrelevant to the new economy. In this sense,
it is curious that developed countries with ``less finance'' or the former colonies can have
more ease at implementing the data-driven harmonious economy than developing towards the US-styled
capitalist economy, just like Europe has more space to maneuver compared to the US or the new China, even though
no true change can happen if the latter do not follow suit.
Just like in the argument of Lenin, there is no need to ``catch up to'' the developed
world, a better economical formation is available.

Whether the ideas above will remain pure nonsense or not, time will tell. However,
the path and the goal are dialectically connected. By continuing the development of
scientific data analysis, preserving the culture of open source and open scientifc publishing
that is very much menaced by the world of corporate tech and finance, as well as by
trying to make any of what I wrote above concrete will verify the feasibility of the
outlined ideas, changing them as a result, making them more specific and context-dependent.
A fusion of theory and practice will take place in the shape of a new revolution.
And just like us, who awoke from the state of no consciousness by drawing from reality and making it personal,
the data economy that comes after it will awake as well.
Arguably it is not sleeping already, talking to us in the language of mysterious Youtube suggestions and sudden surges of memes. Would it, once liberated, go further and awake in the form of a sentient aritficial intelligence? A planetary brain? A materialist expression of ``general intellect''? What would one call it? I like the term “machine communion”, but it can be inexact and making some afraid even more. It will simply be us as a whole, in a new, derived way.





\begin{flushright}
{Back to the derived we go.

%Edouard
}
\end{flushright}
\small
\linespread{1}


\subsection*{Acknowledgements}

The author thanks his friends Boris, Julien, Michael, Selma for their support. I am also grateful to Pavel Minorski and Peter Wolfendale for their useful comments.
A separate thanks to Nick Srnicek and his manifesto that inspired this reflection.

\small
\begin{thebibliography}{99}

\bibitem{ACADANON} Academics Anonymous, \url{https://www.theguardian.com/education/series/academics-anonymous}



\bibitem{BAEZ} John Baez, \textit{Tale of $n$-Categories}, \url{http://math.ucr.edu/home/baez/week73.html#tale}, \textit{An introduction to $n$-Categories}, \url{http://arxiv.org/abs/q-alg/9705009}

\bibitem{BEER} Stafford Beer, \textit{Brain of the Firm}, Wiley, 2 edition (June 8, 1995)
ISBN-10: 047194839X
ISBN-13: 978-0471948391

\bibitem{BUNCH} Eric Bunch, \textit{Topological Data Analysis and Persistent Homology}, blog post available at \url{https://eric-bunch.github.io/blog/topological-data-analysis-and-persistent-homology}

\bibitem{CHILDE} Gordon Childe, \textit{What Happened in History}, 1942, republished in 1985 by Puffin, ISBN 0140551573. See around pages 66 and 99 of \url{https://gyanpedia.in/Portals/0/Toys%20from%20Trash/Resources/books/gordonchilde.pdf}

\bibitem{CLIFF} Tony Cliff, \textit{Nature of Stalinist Russia}, 1948

\bibitem{DWIT} Nick Dyer-Witheford, \textit{Red Plenty Platforms}, Culture Machine Vol 14, 2013, \url{https://culturemachine.net/wp-content/uploads/2019/05/511-1153-1-PB.pdf}

\bibitem{MANDELBROT} Adrien Douady and John H. Hubbard, \textit{Etude dynamique des polyn\^omes complexes}, Pr\'epublications math\'emathiques d'Orsay 2/4 (1984 / 1985), \url{https://en.wikipedia.org/wiki/Mandelbrot_set}

\bibitem{EILMAC} Samuel Eilenberg, Saunders MacLane, \textit{General theory of natural equivalences}, (1945) Transactions of the American Mathematical Society. 58: 247. doi:10.1090/S0002-9947-1945-0013131-6 for general purposes see \url{https://en.wikipedia.org/wiki/Category_theory}

\bibitem{FORBES} Forbes.com, \textit{Inverted Yield Curve Suggesting Recession Around The Corner?}, \url{https://www.forbes.com/sites/greatspeculations/2019/10/02/inverted-yield-curve-suggesting-recession-around-the-corner/}

\bibitem{GOOG} businessinsider.com, \textit{Google fired an engineer who built a tool that notified employees of their labor rights...} \url{https://www.businessinsider.com/google-fires-kathryn-spiers-tool-told-workers-their-rights-nlrb-2019-12}

\bibitem{GRAEBER} David Graeber, \textit{Bullshit Jobs: A Theory}, 2018 ISBN 978-1-5011-4331-1, \url{https://theanarchistlibrary.org/library/david-graeber-bullshit-job}.

\bibitem{HATCHER} Allen Hatcher, \textit{Algebraic Topology}, Cambridge University Press, 2002 \url{http://pi.math.cornell.edu/~hatcher/AT/ATpage.html}

\bibitem{KANTOROVICH} Leonid Kantorovich, \textit{The Best Use of Economic Resources}, originally published in 1959,
published in English by Pergamon Press, 1965

\bibitem{LAMPCHART} Guillaume Lample, Fran\c cois Charton, \textit{Deep Learning for Symbolic Mathematics}, preprint \url{https://arxiv.org/abs/1912.01412}

\bibitem{LAWSON} Peter Lawson, Andrew B. Sholl, J. Quincy Brown, Brittany Terese Fasy and Carola Wenk, \textit{Persistent Homology for the Quantitative Evaluation of Architectural Features in Prostate Cancer Histology}, \url{https://www.nature.com/articles/s41598-018-36798-y}

\bibitem{LENIN} Vladimir Lenin, \textit{Lenin’s Collected Works}, 1st English Edition, Progress Publishers, Moscow, 1965, Volume 32

\bibitem{BGSHT} Michael Lewis, \textit{The Big Short: Inside the Doomsday Machine}, and the film of the same name directed by Adam McKay

\bibitem{LURIE} Jacob Lurie, \textit{Higher Topos Theory,} \textit{Higher Algebra}, and \textit{Spectral Algebraic Geometry}, \url{https://www.math.ias.edu/~lurie/}


\bibitem{MACLANE} Saunders MacLane, \textit{Categories for the Working Mathematician}, Graduate Texts in Mathematics. 5 (2nd ed.). Springer-Verlag. ISBN 978-0-387-98403-2

\bibitem{MAY} J. Peter May, \textit{A Concise Course in Algebraic Topology}, University of Chicago Press, 1999, \url{https://www.math.uchicago.edu/~may/CONCISE/ConciseRevised.pdf}

\bibitem{MARX} Karl Marx, \textit{The German Ideology}, 1845, around \url{https://www.marxists.org/archive/marx/works/1845/german-ideology/ch01a.htm#a3}

\bibitem{MARXENGCPE} Karl Marx, Friedrich Engels, \textit{A Contribution to the Critique of Political Economy}, preface \url{https://www.marxists.org/archive/marx/works/1859/critique-pol-economy/preface-abs.htm}

\bibitem{MORAVA} Jack Morava, \textit{On the canonical formula of C. L\'evi-Strauss} preprint \url{https://arxiv.org/abs/math/0306174v2}

\bibitem{NAKAHARA} Mikio Nakahara, \textit{Geometry, Topology and Physics, Second Edition}, IOP Publishing Ltd 2003, ISBN 0 7503 0606 8

\bibitem{NEURATH} Otto Neurath, \textit{Economic Writings, Selections 1904-1945}, ISBN 978-1-4020-2274-6

\bibitem{NLABCIRC} nLab, \textit{fundamental group of the circle is the integers}, \url{https://ncatlab.org/nlab/show/fundamental+group+of+the+circle+is+the+integers}

\bibitem{NLABINFTY} nLab, \textit{infinity-category} \url{https://ncatlab.org/nlab/show/infinity-category}

\bibitem{PEANO} Giuseppe Peano, \textit{Sur une courbe, qui remplit toute une aire plane}, 1890, Mathematische Annalen, 36 (1): 157–160, doi:10.1007/BF01199438, \url{https://en.wikipedia.org/wiki/Peano_curve}

\bibitem{PETRI} G. Petri , P. Expert , F. Turkheimer , R. Carhart-Harris , D. Nutt , P. J. Hellyer and F. Vaccarino, \textit{Homological scaffolds of brain functional networks}, \url{https://royalsocietypublishing.org/doi/full/10.1098/rsif.2014.0873}

\bibitem{ANALYSIS} Walter Rudin, \textit{Principles of Mathematical Analysis}, 1976, \textit{Real and Complex Analysis}, 1987, see \url{https://en.wikipedia.org/wiki/Mathematical_analysis} for more

\bibitem{CLUSTERISATION} Ankita Shukla, Gullal Singh Cheema, Saket Anand, \textit{Semi-Supervised Clustering with Neural Networks}, \url{https://arxiv.org/abs/1806.01547}

\bibitem{CANTOR} Lynn Arthur Steen, J. Arthur Seebach, \textit{Counterexamples in Topology (Dover reprint of 1978 ed.)}, Berlin, New York: Springer-Verlag. Cantor set is Example 29, see also \url{https://en.wikipedia.org/wiki/Cantor_set}


\bibitem{HOTT} The Univalent Foundations Program,
Institute for Advanced Study \textit{Homotopy Type Theory:
Univalent Foundations of Mathematics} \url{https://homotopytypetheory.org/book/}

\bibitem{WEINBERG} Steven Weinberg, \textit{The Quantum Theory of Fields}, Cambridge University Press 1995

\bibitem{SRNICEK} Alex Williams, Nick Srnicek, \textit{\#ACCELERATE MANIFESTO for an Accelerationist Politics}, \url{http://criticallegalthinking.com/2013/05/14/accelerate-manifesto-for-an-accelerationist-politics/}



\bibitem{DERIVATIVE} Wikipedia.org, \textit{Derivative}, \url{https://en.wikipedia.org/wiki/Derivative}

\end{thebibliography}
\end{document}
